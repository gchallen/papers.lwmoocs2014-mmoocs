Massive open online courses (MOOCs) have the potential to transform teaching
and learning, and their impact is already being felt on university campuses
and in popular culture. To meet the diverse learning needs of students at
many institutions, MOOCs must encourage collaboration, not confrontation,
with the professors using them. Today's online courses, typically created and
controlled by small groups of faculty, both fail to foster this collaboration
and also miss the opportunity to leverage the experience and experimental
capacity present in classrooms across the country. A variety of instructors
teaching a variety of different students in a variety of different ways
represents an incredible education resource and living laboratory that MOOCs
must find a way to harness, not suppress.
