\documentclass[10pt]{whitepaper}
\usepackage{url,graphicx,multirow,hyperref}
\hypersetup{
  bookmarks=true,
  unicode=true,
  pdftoolbar=true,
  pdfmenubar=true,
  pdffitwindow=true,
  pdfstartview={FitV},
  pdftitle={},
  pdfauthor={},
  pdfnewwindow=true,
  colorlinks=false,
  pdfdisplaydoctitle=true,
  pdfborder={0 0 0},
  pdfinfo={
    Title={Enabling MOOC Collaborations Through Modularity},
    Author={Geoffrey Challen},
  },
}
\usepackage[absolute]{textpos}

\setlength{\TPHorizModule}{1in}
\setlength{\TPVertModule}{1in}
\textblockorigin{1in}{1in}

\usepackage[all]{hypcap}

\paperheight 11in
\paperwidth 8.5in

\input{.xxxnote}
\input{.blue}
% 16 Nov 2010 : GWA : Any special macros or other stuff for this particular
%               whitepaper go here.
\newcommand{\opsclass}{\href{http://www.ops-class.org}{\texttt{ops-class.org}}}

\pagestyle{plain}


\def\thetitle{Enabling MOOC Collaborations Through Modularity}
\def\shorttitle{Modular MOOCs}
\def\theauthors{Geoffrey Challen}
\def\shortauthors{Challen}
\def\submissiondate{\today}
\def\whitepapername{Learning with MOOCs}

\begin{document}
\pagestyle{whitepaper}
\thispagestyle{emptywhitepaper}
\pagenumbering{arabic}

\chapter{\thetitle}

Massive open online courses (MOOCs) have the potential to transform teaching
and learning by generating, analyzing, and acting on massive data sets that
are much larger than any instructor at a traditional university would usually
have access to. But these large data sets are not enough. Determining what
works requires that we simultaneously encourage experimentation and harness
instructional diversity already present in today's traditional university
classrooms, while leveraging MOOCs to generate well-structured data for
analysis. Big educational data sets have to be more than just big---they must
be \textit{diverse}, allowing different methods of teaching and learning to
be evaluated in the context of different expectations and student
populations. A single online course produced by a small set of faculty cannot
produce this diversity. Engaging many faculty and students at different
institutions is necessary.

The tension between MOOCs and traditional universities is palpable: In a
well-publicized rejection of Harvard's popular EdX course on social justice,
philosophy faculty at San Jose State University raised concerns over the
ability of online content to integrate into their existing courses:

\begin{quote}

When a university such as ours purchases a course from an outside vendor, the
faculty cannot control the design or content of the course; therefore we
cannot develop and teach content that fits with our overall curriculum and is
based on both our own highly developed and continuously renewed competence
and our direct experience of our students' needs and abilities.

\end{quote}

As systems engineers, we see this complaint as expressing not a fundamental
limitation of MOOCs, but a design flaw in how they are currently built:
\textbf{today's MOOCs are not modular}. Modularity has long been appreciated
when designing and implementing computer systems. Monolithic ``blobs'' of
code, usually produced by a single developer to quickly solve a particular
problem, are difficult to improve, test, and maintain, and are unlikely to be
used by anyone but their creator. All of today's major online education
providers, including Coursera, Udacity, and the edX platform, provide
monolithic MOOC content: entire courses that are packaged and offered as an
indivisible unit. This flaw limits their ability to partner with faculty at
other institutions, and to generate large and diverse data sets.

Since today's MOOCs suffer from a common software design problem, we propose
to apply a standard software engineering solution: modularization. Modular
MOOCs break content into small \textit{modules} all sharing a common
structure. Individual modules can be tested and improved in isolation and
their impact on the entire class determined. As the goals and expectations of
the class for which they were originally created change, and as the content
itself changes, it is more likely that the functionality provided by modules
will continue to remain relevant, while adding new modules keeps the course
current.

Most importantly, modularization encourages the interinstitution
collaboration that today's monolithic MOOCs struggle to achieve. Modules
allow faculty to retain full control over course content and structure, with
the ability to add, remove, alter and reorder modules in ways expressing
their understanding of their students' needs and abilities, all while
producing experimental data that benefits the original course. When faculty
provide more effective replacements for existing modules, everyone benefits.
And if one faculty member discovers an effective sequence of modules, all
participating faculty can decide whether to adjust their own courses to
incorporate this insight.

We propose to explore the design and construction of \textit{modular MOOCs
(mMOOCs)}, which exploit modularity to provide the control traditional
faculty desire while providing the diverse and well-structure data sets
required to answer fundamental questions about teaching and learning.
Together, the co-PIs will design and build a mMOOC to teach operating systems
online. Then, they will recruit faculty from other institutions to foster
vibrant multi-institutional collaborations in education, just as we currently
do in research.

\end{document}
